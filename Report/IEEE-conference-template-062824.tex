\documentclass[conference]{IEEEtran}
%\IEEEoverridecommandlockouts
% The preceding line is only needed to identify funding in the first footnote. If that is unneeded, please comment it out.
%Template version as of 6/27/2024

\usepackage{cite}
\usepackage{amsmath,amssymb,amsfonts}
\usepackage{algorithmic}
\usepackage{graphicx}
\usepackage{textcomp}
\usepackage{xcolor}
\def\BibTeX{{\rm B\kern-.05em{\sc i\kern-.025em b}\kern-.08em
    T\kern-.1667em\lower.7ex\hbox{E}\kern-.125emX}}
\begin{document}

\title{Conference Paper Title*\\
{\footnotesize \textsuperscript{*}Note: Sub-titles are not captured for https://ieeexplore.ieee.org  and
should not be used}
\thanks{Identify applicable funding agency here. If none, delete this.}
}

\author{\IEEEauthorblockN{1\textsuperscript{st} Danyal Mirza}
\IEEEauthorblockA{\textit{dept. name of organization (of Aff.)} \\
\textit{name of organization (of Aff.)}\\
City, Country \\
email address or ORCID}
\and
\IEEEauthorblockN{2\textsuperscript{nd} Given Name Surname}
\IEEEauthorblockA{\textit{dept. name of organization (of Aff.)} \\
\textit{name of organization (of Aff.)}\\
City, Country \\
email address or ORCID}
\and
\IEEEauthorblockN{3\textsuperscript{rd} Given Name Surname}
\IEEEauthorblockA{\textit{dept. name of organization (of Aff.)} \\
\textit{name of organization (of Aff.)}\\
City, Country \\
email address or ORCID}
\and
\IEEEauthorblockN{4\textsuperscript{th} Given Name Surname}
\IEEEauthorblockA{\textit{dept. name of organization (of Aff.)} \\
\textit{name of organization (of Aff.)}\\
City, Country \\
email address or ORCID}
\and
\IEEEauthorblockN{5\textsuperscript{th} Given Name Surname}
\IEEEauthorblockA{\textit{dept. name of organization (of Aff.)} \\
\textit{name of organization (of Aff.)}\\
City, Country \\
email address or ORCID}
\and
\IEEEauthorblockN{6\textsuperscript{th} Given Name Surname}
\IEEEauthorblockA{\textit{dept. name of organization (of Aff.)} \\
\textit{name of organization (of Aff.)}\\
City, Country \\
email address or ORCID}
}

\maketitle

% For citations use \cite{x}, where x is defined as \bibitem{x} in references

\begin{abstract}
Linnea + Danyal
\end{abstract}

\begin{IEEEkeywords}
component, formatting, style, styling, insert.
\end{IEEEkeywords}

\section{Introduction}
Linnea + Danyal

\section{Methodology}
Linnea + Danyal

\subsection{Overall system design}
Linnea + Danyal

\subsection{User perception sub-system}
Linnea + Danyal

\subsection{Interaction sub-system} %Walter + Danyal
This section discusses the second subsystem of Bart the bartender. 
It goes over the high-level design and its implementation. 
Afterwards it evaluates the obtained results

\subsubsection{Design}
The interaction sub-system in this project consists of the virtual furhat, Gemini and a connection to the video server.
Furhat is used as the interface between the system and the user and Gemini is used to generate a response to the input of the user.
The response is based on what the user says and what emotion is detected in the face of the user at the beginning of the conversation.
This response is then send to Furhat so that it can give a reply to the user.
This process gets repeated until the right drink is found for the user.

\subsubsection{Implementation}
The typical flow of this subsystem is as follows: 
A connection is set up between this subsystem and the video server that was described in the previous section.
Once this connection is set up, Bart greets the user and asks what kind of cocktail the user wants.
From this moment, the subsystem gathers the detected emotions of the user, until the user is done with speaking.
From these emotions, we determine the dominant emotion of the user.
When the user finishes their reply to Bart, we combine the dominant emotion with the reply of the user and send it to Gemini.
Gemini is a LLM that is instructed to act like a bartender that recommends cocktails that fit the mood of the customer.
We decided to use Gemini because it is free for people with a google account up to a rate limit.
Our application does not use up a lot of tokens, so we should not reach this rate limit.
Once Gemini has generated a response, it is forwarded to Bart.
Then, Bart is instructed to say the response.
After Bart is done with talking, it listens again for input of the user. 
The input of the user is again combined with the dominant emotion and sent to Gemini.
This cycle continues until the user decides on a cocktail.
After this the whole cycle repeats from the beginning, assuming there is a new customer.

% The interaction sub-system mostly resides in a file called \textit{client\_receive.py}.
% This file first sets up the connection to the video server and after that it listens for messages from the server.
% In our case, messages from the server contain information about the faces in the frame.
% This includes emotion and postion of the face.
% The system decides at the beginning of the conversation with the user what the most dominant emotion of the user is.
% It does this by gathering all detected emotions until the user gives their first input.
% While \textit{client\_receive.py} listens to messages from the server, it also listening for input of the user.
% If the user is done with talking, their message is combined with most dominant emotion .
% This prompt is then given to Gemini, which creates a fitting response.
% We chose Gemini for this project because it is free to use with a Google account and it has a rate limit that we will most likely not reach.
% To make sure Gemini generates a fitting response, it is given a system instruction that precisely describes its role as a bartender.
% The response that is generated by Gemini is then forwarded to furhat so it can reply to the input of the user.
% After this is done, furhat starts listening to the user again and this process is repeated.
% Once Bart finds a fitting cocktail it says `Here you go' and afterwards listens to input from a new user.

\subsubsection{Result}
In the project proposal, we described several objectives that we expect our bartender to fullfill.
We can test our product by confirming it adheres to the set objectives.

One of the objectives is that our product gives recommendations to the user, based on the user's emotional state.
We add the mood of the user to the prompt that we send to Gemini.
Because of this, Gemini will take both the emotion of the user as well as the user's speech into account when generating a response.

Another objective is that Bart should take the user's preferences into account. 
We achieve this by letting Gemini generate questions that explores the needs and preferences of the user.
Gemini incorporates current and previous answers of this user into the recommendation.

The last objective is that the conversation with the system should be pleasant.
This is a more subjective goal, but we try to make the conversation as pleasant as possible by continuously adjusting the prompts given to Gemini.
By doing this, we hope to make the conversation as pleasant as possible.

% - works well in general, it finds a cocktail based on the input of the user relatively quickly
% objectives from specification:
% - come up with recommendations based on emotional state -> reflects on current emotional state of user
% - take the preferences of the client into account -> automatically happens due to prompting of llm
% - make sure the conversation is pleasant -> subjective, hard to tell (conduct interviews etc.)

\section{General Discussion}
Adria
\subsection{Overall Pipeline}
Adria
\subsection{Challenges}
Adria
\subsection{Use of ChatGPT and other tools}
Adria
\subsection{Ethical Issues}
Adria



\section*{Conclusion}
Adria

\begin{thebibliography}{00}
\bibitem{b1} G. Eason, B. Noble, and I. N. Sneddon, ``On certain integrals of Lipschitz-Hankel type involving products of Bessel functions,'' Phil. Trans. Roy. Soc. London, vol. A247, pp. 529--551, April 1955.

\end{thebibliography}

\end{document}
