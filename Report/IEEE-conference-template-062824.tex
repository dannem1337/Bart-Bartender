\documentclass[conference]{IEEEtran}
%\IEEEoverridecommandlockouts
% The preceding line is only needed to identify funding in the first footnote. If that is unneeded, please comment it out.
%Template version as of 6/27/2024

\usepackage{cite}
\usepackage{amsmath,amssymb,amsfonts}
\usepackage{algorithmic}
\usepackage{graphicx}
\usepackage{textcomp}
\usepackage{xcolor}
\def\BibTeX{{\rm B\kern-.05em{\sc i\kern-.025em b}\kern-.08em
    T\kern-.1667em\lower.7ex\hbox{E}\kern-.125emX}}
\begin{document}

\title{Conference Paper Title*\\
{\footnotesize \textsuperscript{*}Note: Sub-titles are not captured for https://ieeexplore.ieee.org  and
should not be used}
\thanks{Identify applicable funding agency here. If none, delete this.}
}

\author{\IEEEauthorblockN{1\textsuperscript{st} Danyal Mirza}
\IEEEauthorblockA{\textit{dept. name of organization (of Aff.)} \\
\textit{name of organization (of Aff.)}\\
City, Country \\
email address or ORCID}
\and
\IEEEauthorblockN{2\textsuperscript{nd} Given Name Surname}
\IEEEauthorblockA{\textit{dept. name of organization (of Aff.)} \\
\textit{name of organization (of Aff.)}\\
City, Country \\
email address or ORCID}
\and
\IEEEauthorblockN{3\textsuperscript{rd} Given Name Surname}
\IEEEauthorblockA{\textit{dept. name of organization (of Aff.)} \\
\textit{name of organization (of Aff.)}\\
City, Country \\
email address or ORCID}
\and
\IEEEauthorblockN{4\textsuperscript{th} Given Name Surname}
\IEEEauthorblockA{\textit{dept. name of organization (of Aff.)} \\
\textit{name of organization (of Aff.)}\\
City, Country \\
email address or ORCID}
\and
\IEEEauthorblockN{5\textsuperscript{th} Given Name Surname}
\IEEEauthorblockA{\textit{dept. name of organization (of Aff.)} \\
\textit{name of organization (of Aff.)}\\
City, Country \\
email address or ORCID}
\and
\IEEEauthorblockN{6\textsuperscript{th} Given Name Surname}
\IEEEauthorblockA{\textit{dept. name of organization (of Aff.)} \\
\textit{name of organization (of Aff.)}\\
City, Country \\
email address or ORCID}
}

\maketitle

% For citations use \cite{x}, where x is defined as \bibitem{x} in references

\begin{abstract}
Linnea + Danyal
\end{abstract}

\begin{IEEEkeywords}
component, formatting, style, styling, insert.
\end{IEEEkeywords}

\section{Introduction}
Linnea + Danyal

\section{Methodology}
Linnea + Danyal

\subsection{Overall system design}
Linnea + Danyal

\subsection{User perception sub-system}
Linnea + Danyal

\subsection{Interaction sub-system} %Walter + Danyal
This section discusses the second subsystem of Bart the bartender. 
It goes over the high-level design and its implementation. 
Afterwards it evaluates the obtained results

\subsubsection{Design}
The interaction sub-system in this project consists of the virtual furhat, Gemini and a connection to the video server.
Furhat is used as the interface between the system and the user and Gemini is used to generate a response to the input of the user.
The response is based on what the user says and what emotion is detected in the face of the user.
This response is then send to Furhat so that it can give a reply to the user.
This process gets repeated until the right drink is found for the user.

\subsubsection{Implementation}
The interaction sub-system mostly resides in a file called \textit{client\_receive.py}.
This file first sets up the connection to the video server and after that it constantly listens for messages from the server.
In our case, messages from the server contain information about the faces in the frame.
This includes emotion and postion of the face.
While \textit{client\_receive.py} listens to messages from the server, it also listening for input of the user.
If the user is done with talking, their message is combined with emotion from the latest message from the video server.
This prompt is then given to Gemini, which creates a fitting response.
We chose Gemini for this project because it is free to use with a Google account and it has a rate limit that we will most likely not reach.
To make sure Gemini generates a fitting response, it is given a system instruction that precisely describes its role as a bartender.
The response that is generated by Gemini is then forwarded to furhat so it can reply to the input of the user.
After this is done, furhat starts listening to the user again and this process is repeated.
Once Bart finds a fitting cocktail it says `Here you go' and afterwards listens to input from a new user.

\subsubsection{Result}
- works well in general, it finds a cocktail based on the input of the user relatively quickly
objectives from specification:
- come up with recommendations based on emotional state -> reflects on current emotional state of user
- take the preferences of the client into account -> automatically happens due to prompting of llm
- make sure the conversation is pleasant -> subjective, hard to tell (conduct interviews etc.)

\section{General Discussion}
Adria
\subsection{Overall Pipeline}
Adria
\subsection{Challenges}
Adria
\subsection{Use of ChatGPT and other tools}
Adria
\subsection{Ethical Issues}
Adria



\section*{Conclusion}
Adria

\begin{thebibliography}{00}
\bibitem{b1} G. Eason, B. Noble, and I. N. Sneddon, ``On certain integrals of Lipschitz-Hankel type involving products of Bessel functions,'' Phil. Trans. Roy. Soc. London, vol. A247, pp. 529--551, April 1955.

\end{thebibliography}

\end{document}
